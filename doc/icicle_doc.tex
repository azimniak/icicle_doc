\documentclass[11pt]{article}
\pdfoutput=1
\usepackage[utf8]{inputenc}

\usepackage{enumitem}

%\usepackage[colorlinks=true,allcolors=blue]{hyperref}

\usepackage{indentfirst}
\setlength{\parindent}{0.4cm}
\usepackage{microtype}
\usepackage[hmarginratio=1:1,top=24mm,bottom=20mm,left=17mm,right=17mm,columnsep=20pt]{geometry}
\usepackage[pdftex]{graphicx}
\usepackage{caption}

\usepackage{authblk}
\author{Anna Zimniak}
\affil{Institute of Geophysics, Faculty of Physics, University of Warsaw, Poland} 


\title{Kinematic model documentation}

\begin{document}

\maketitle

\section{Introduction}

The kinematic model called 'icicle' simulates a convective 2 - dimensional cloud in a domain of a size 1500~m in horizontal and 1500~m in vertical. It uses two libraries:
\begin{itemize}
	\item \textit{libcloudph++} - which represents all microphysical processes;
	\item \textit{libmpdata++} - which controlls advection.
\end{itemize}

Input arguments and output parameters of the model are described in following sections.

\section{Arguments}

\subsection{Grid attributes}

\begin{itemize}
\item \textbf{nx} : int (default = 76);\\ grid cell count in horizontal
\item \textbf{nz} : int (default = 76);\\ grid cell count in vertical

A size of every single cell is given by $\frac{1500}{\mbox{nx}-1}\mbox{[m]  \scriptsize{x} }\frac{1500}{\mbox{ny}-1} \mbox{[m]}$.
\end{itemize}

\subsection{Output parameters}

\begin{itemize}
  \item \textbf{outdir} : string (required); \\ output catalog name
  \item \textbf{outfreq} : int (required); \\ output frequency (time gap between outputted points in number of time steps)
\end{itemize}

\subsection{Simulation parameters}

\begin{itemize}
\item \textbf{nt} : int (default = 3600);\\ number of timesteps
\item \textbf{spinup} : int (default = 2400);\\ number of initial timesteps during which collision - coalescence processes are turned off
\item \textbf{adv\_serial} : bool (default = false);\\ if 'true' - advection is computed on a single thread, \\if 'false' - advection is computed on multiple threads. 
\item \textbf{relax\_th\_rv} : bool (default = true);\\ if 'true' - potential temperature and water vapour mass mixing ratio tend to relax to initial condition during the simulation, \\ if 'false' - these variables change with condensation and coalescense and rain removes water vapour from the domain. To achieve a stationary situation this parameter must be set to 'true'.
\item \textbf{micro} : str (required);\\ defines a method for microphysics computation. Valid options are: 'blk\_1m', 'blk\_2m' and 'lgrngn'.
 
Next arguments of the model depend on the choice of the \textbf{micro} option.

\textbf{'blk\_1m'} : a single-moment microphysics scheme. Arguments for this scheme are: 
\begin{itemize}[label=$\bullet$]
\item \textbf{cond} : bool (default = true); \\ cloud water condensation
\item \textbf{cevp} : bool (default = true); \\ cloud water evaporation
\item \textbf{revp} : bool (default = true); \\ rain water evaporation
\item \textbf{conv} : bool (default = true); \\ autoconversion of cloud water into rain
\item \textbf{accr} : bool (default = true); \\ cloud water collection by rain
\item \textbf{sedi} : bool (default = true); \\ rain water sedimentation.
\end{itemize}


\textbf{'blk\_2m'} : a double-moment microphysics scheme. Arguments for this scheme are:
\begin{itemize}[label=$\bullet$]
\item \textbf{acti} : bool (default = true); \\ cloud droplet activation
\item \textbf{cond} : bool (default = true); \\ cloud water condensation
\item \textbf{accr} : bool (default = true); \\ cloud water collection by rain
\item \textbf{acnv} : bool (default = true); \\ autoconversion of cloud water into rain
\item \textbf{sedi} : bool (default = true); \\ rain water sedimentation.
\end{itemize}

\textbf{'lgrngn'} : a particle-based scheme (aka Lagrangian scheme). It allows to track particles/ aerosol/ cloud droplets/ raindrops properties throught the entire simulation. Arguments for this scheme are:
\begin{itemize}[label=$\bullet$]
\item \textbf{backend} : string (required); \\ specification of a device, on which computations with microphysics library are conducted. Valid options are: \\'CUDA' - graphics card, \\'OpenMP' - multithreaded on the processor, \\ 'serial' - single-threaded on the processor. 
\item \textbf{async} : bool (default = true, ignored if backend is not set to 'CUDA'); \\if 'true' - computations for advection on the processor and computations for microphysics on the graphics card are being conducted simultaneously,\\
if 'false' - these computations are being conducted in sequence.
\item \textbf{sd\_conc} : int (required); \\ number of super-droplets per grid cell used in the particle-based microphysics scheme
\item \textbf{adve} : bool (default = true); \\
particle advection
\item \textbf{sedi} : bool (default = true);\\
particle sedimentation
\item \textbf{cond} : bool (default = true);\\
particle condensational growth
\item \textbf{coal} : bool (default = true);\\
particle collisional growth
\item \textbf{chem\_dsl} : bool (default = false);\\
dissolving trace gases
\item \textbf{chem\_dsc} : bool (default = false);\\
dissociation
\item \textbf{chem\_rct} : bool (default = false);\\
chemical reactions
\item \textbf{sstp\_cond} : int (default = 1);\\
number of substeps for condensation
\item \textbf{sstp\_coal} : int (default = 1); \\
number of substeps for coalescence
\item \textbf{sstp\_chem} : int (default = 1); \\
number of substeps for chemistry
\item \textbf{out\_dry} : string (default = "0:1$|$0");\\
dry radius ranges and moment numbers used to generate dry spectra. A way in which this argument has to be constructed in the following way:\\

"left1:right1$|$n1,n2,n3...;left2:right2$|$n1,n2,n3...;..."\\

'Left' and 'right' denote left hand side and right hand side edges of the spectrum bins in meters. n1, n2, n3 etc. after the vertical line are numbers of desired moments of the spectrum. If more bins are wanted, they shoud be specified in the same way after semicolon.

Default argument will generate one 0-th moment spectrum bin, which edges are specified from 0 to 1 m (the range is so wide to cover all sizes of particles).

\item \textbf{out\_wet} : string (default = ".5e-6:25e-6$|$0,1,2,3;25e-6:1$|$0,3,6")\\
wet radius ranges and moment numbers used to generate wet spectra. Construction of this argument is the same as in 'out\_dry'.

Default argument will generate 7 output spectrum bins:\\
- 0-th, 1-st, 2-nd and 3-th spectrum moments for the radius range 0.5 - 25 $\mu$m;

- 0-th, 3-rd and 6-th spectrum moments for particles larger than 25 $\mu$m.

\end{itemize}

\end{itemize}

\section{Output}

The output files are put in the catalog 'outdir' (specified by the user). Output catalog consists of several HDF5 files:

\begin{itemize}
\item 'const.h5' - contains parameters constant through the entire simulation;
\item 'timestep$<$number$>$.h5' - contains parameters for the specific time.
\end{itemize}

'timestep' files are printed with a time gap specified in the 'outfreq' argument. 'number' in the file name denotes actual timestep.

The content of each HDF5 output file can be viewed in a terminal by using the 'h5dump' command.

For Lagrangian microphysics scheme in 'const.h5' file there are following parameters:

\begin{itemize}
\item \textbf{G} - dry air density [kg m\textsuperscript{-3}]. It is an array of a size nx x nz.
\item \textbf{T} - time [s]. It is a list of outputed times. This dataset has an attribute \textbf{dt}, which is a timestep used in the simulation. It's value is set to 1 [s]. 
\item \textbf{X} - number of the cell in hirizontal. It is an nx~x~nz array.
\item \textbf{Y} - number of the cell in vertical. It is an nx~x~nz array.
\end{itemize}

All output variables in 'timestep' files consist of arrays of a size nx~x~nz, so that each cell in the domain has a value attributed. In each 'timestep' file there are following variables:

\begin{itemize}
\item \textbf{rd\_rng$<$number\_of\_range$>$\_mom$<$number\_of\_moment$>$} - dry radius spectrum bin for range and moment specified in the input argument. A value of the moment is given per 1~kg of dry air. Ranges are enumerated in the same order as they were specified in the input. 

\item \textbf{rw\_rng$<$number\_of\_range$>$\_mom$<$number\_of\_moment$>$} - wet radius spectrum bin for range and moment specified in the input argument. A value of the moment is given per 1~kg of dry air. Ranges are enumerated in the same order as they were specified in the input.  

\item \textbf{rv} - water vapour mixing ratio [kg/kg],

\item \textbf{sd\_conc} - super droplet concentration [number per grid cell],

\item \textbf{th} - dry potential temperature [K].

\end{itemize}

\end{document}
